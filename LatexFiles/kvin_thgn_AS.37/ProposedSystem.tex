\section{Proposed system}
The proposed system is made of a system in two parts. The first part is the client, running on a regular computer, the second part is a server running on a centralised computer, available to all running clients through the internet.
\subsection{Functional requirements}
The calendar system supports 2 different kinds of users:
\begin{itemize}
\item Users who can manage entries in their calendar and manage their account.
\end{itemize}
\subsection{Nonfunctional requirements}

\begin{center}
    \begin{tabular}{ | l | p{10cm} |}
    \hline
    Category & Nonfunctional requirements \\ \hline
    Usability & The UI of the client must resemble a paper calendar, as to ease learning process.\\ \hline
    Reliability & Loss of connection to the server must only cut off functions requiring connection, but not access to information already loaded.\\ \hline
    Performance & The server must support multiple clients at once (e.g., 25).
	The communication between client and server, when updating a server should be fast even with a low bandwidth connection. \\ \hline
	Supportability & ... \\ \hline
	Implementation & The system should be implemented in C\# and work on all newer versions of Windows \\ \hline
	Operation & ... \\ \hline
	Legal & ... \\ \hline
    \end{tabular}
\end{center}
\subsection{System models}

\subsubsection{Scenarios}
\textbf{Scenario name:} calendarEntryMeeting

Participating actors: John, User.

\textbf{Flow of events:}
\begin{itemize}
\item John is invited to meeting the 21st of July 2015. To remember it he wants to make an entry in his calendar. He accesses the create new entry function in the calendar system.
\item John plots the information of the meeting and the date/time. He confirms the input and waits for the calendar to create the entry.
\item The calendar system tells John that the entry has successfully been created.
\end{itemize}